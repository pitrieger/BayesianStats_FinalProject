\section{Appendix}

\setcounter{table}{0}
\renewcommand{\thetable}{A\arabic{table}}
\setcounter{figure}{0}
\renewcommand{\thefigure}{A\arabic{figure}}


\subsection{Stan Code}
\begin{lstlisting}[language=Stan, caption = {Stan Model Code}, captionpos = t, label = {code:stan}]
data {
    int<lower=0> n_event; // # uncensored obs         
    int<lower=0> n_censor; // # censored obs
    int<lower=0> p; // # covariates excluding intercept
    matrix[n_event,p] X_event; // uncensored obs design matrix 
    matrix[n_censor,p] X_censor; // censored  -----"----------
    vector<lower=0>[n_event] T_event; // uncensored survival times
    vector<lower=0>[n_censor] T_censor; // lower bound of censored
    survival times = censor time                  
}
parameters {
    real alpha;
    vector[p] beta;                                     
}
model {
    vector[n_event] lambda_event = exp(alpha + X_event * beta);
    vector[n_censor] lambda_censor = exp(alpha + X_censor * beta);
    alpha ~ normal(0, 100);
    beta ~ normal(0, 5);
    target += exponential_lpdf(T_event | lambda_event); 
    target += exponential_lccdf(T_censor | lambda_censor);  
}
\end{lstlisting}


\subsection{Zero-Inflated Poisson Regression}
Since the interest is more on the association between survival time and the share of female ministers in the cabinet, the problem can also be approached differently, i.e. outside the survival modelling framework. Typically it is roughly known when the next regular elections will be held. At the start of each government, we can thus already tell what the maximum number of possible days in office for that government will be. Denoting this maximum as $m_i$, we can compute the number of days that the government fell short of reaching that maximum as 

\begin{equation*}
    \widetilde{Y}_i = m_i - Y_i.
\end{equation*}

We thus have a count variable $\widetilde{Y}_i$ that can be modelled as a zero-inflated Poisson random variable. The likelihood of this model is therefore given by 
\begin{equation*}
    \mathcal{L}_i(\theta) = \begin{cases}
    \pi_i + (1-\pi_i) e^{-\lambda_i} \lambda_i^{0} & \text{, $\tilde{y}_i = 0$} \\
    (1-\pi_i) \frac{e^{-\lambda_i} \lambda_i^{y_i}}{\tilde{y}_i!} & \text{, otherwise}
    \end{cases}
\end{equation*}

where $\bm\theta = (\bm\pi, \bm\lambda)$ can be defined via logistic and log link functions, respectively, such that 

\begin{align*}
    \bm\pi &= (1 + \exp(-(\alpha + \bm{X} \bm\beta)))^{-1} \\
    \bm\lambda &= \exp(\nu+ \bm{Z} \bm\gamma)/ 
\end{align*}