KK20:
"
Given the importance of cabinet stability for the performance of the political system, a comprehensive set of literature aims to explain the survival and termination of governments. These studies can be summarised along three traditions: Firstly, game-theoretic research explains cabinet stability as a consequence of the rational choices of coalition parties (Diermeier & Stevenson, 1999; Lupia & Strøm, 1995). Secondly, critical events such as scandals, crises, or international conflicts have been highlighted as determinants of government survival (Browne et al., 1984; Frendreis et al., 1986). Thirdly, the ‘attributes’ approach reveals how a broad variety of cabinet characteristics affect their likelihood of lasting. The institutional setting within which a government operates can influence cabinet stability. The power of the head of state (Strøm & Swindle, 2002) or the prime minister (Schleiter & Morgan-Jones, 2009) to dissolve parliament is particularly relevant. The party composition of governments is a second attribute explaining survival. Having a majority in parliament (see e.g., Saalfeld, 2008), being a minimal winning cabinet (see e.g., Laver, 1974; Saalfeld, 2008), and having ideological compactness within the government (see e.g., Saalfeld, 2008; Warwick, 1979), all have positive effects on cabinet stability.

The three sets of literature agree that conflicts within the government constitute a key explanation for government termination. Once in office, policy-seeking actors will have disputes about concrete policy decisions. Dissent might emerge between coalition partners or ministers from the same party. If conflicts cannot be settled, coalition governments might break apart or parliaments might dissolve single-and multi-party governments to avoid deadlocks (Diermeier & Stevenson, 1999). We argue that the likelihood for disputes to emerge, escalate, and lead to government dissolution, also depends on the leadership styles of the politicians in cabinet. The presence of a head of government and ministers who adopt consensual and compromise-oriented – as opposed to hierarchical and confrontational – strategies, reduces the risk for internal conflicts. This, in turn, increases the odds of government survival. Since women’s leadership style tends to be characterised by higher levels of collaboration than men’s, we hypothesise that the presence of a female prime minister and female ministers positively impacts cabinet stability.2

Women favour solving dissent and conflict through collaborative and compromise-oriented strategies, while men tend to opt for hierarchical and confrontational plans of action (Kellerman et al., 2007; March & Weil, 2005; Norris, 1996). Women’s leadership style has also been described as ‘democratic and consensual’ (Campus, 2013, p. 16), highlighting the fact that they make use of interpersonal ties to find acceptable solutions for all actors involved through persuasion. Two main explanations for these gender differences in leadership style stand side by side: On the one hand, social role theory proposes that individuals adapt to societal expectations about appropriate behaviour, which are shaped by traditional role models, and these norms impede women from following more aggressive conflict resolution strategies (Eagly, 1987). On the other hand, women might develop distinct behavioural patterns to overcome added barriers to success in politics, which involve supporting and collaborating with other actors (Volden et al., 2013).

Three sets of literature lend support to the argument that women are more consensus- and compromise-oriented than men: To begin with, research on women in top management positions shows that female leaders aim to encourage participation in decision-making procedures, and tend to share power, while their male colleagues more frequently engage in top-down decisions (Eagly, 2007; Rosener, 1990). More effective communication skills are a key tool enabling women to successfully implement such strategies (Stanford et al., 1995, p. 15). In the field of international relations, scholars interested in explaining conflicts and their intensity reveal that wars and violence occur less frequently under female leadership (see e.g., Caprioli & Boyer, 2001; Maoz, 2012). Known as the ‘women and peacè hypothesis (Tessler et al., 1999), this second set of literature indicates that women are less belligerent than their male colleagues, are more willing to share resources and take other’s preferences into account – even if this implies not being able to maximize their personal gains. Lastly, scholars of legislative behaviour revealed gender differences between male and female parliamentarians: Women tend to apply democratic and consensual strategies; they invest more time and effort into creating within- and across-party coalitions (Carey et al., 1998; Volden et al., 2013). When asked about their leadership style, female legislators stress their dedication to settling disputes by concessions on each side (Childs, 2000, 2004). Overall, there are various cues that women are more compromise-oriented than men."